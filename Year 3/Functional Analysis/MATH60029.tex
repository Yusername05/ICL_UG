\documentclass{article}
\usepackage{../../style/header}

\begin{document}

\title{Functional Analysis}
\author{Lectured by Pierre Germain \\
Scribed by Yu Coughlin}
\date{Autumn 2025}

\maketitle

\tableofcontents

\subsection{Banach Spaces}

I'll write these notes mostly over $\bR$, everything works similarly over $\bC$ or any complete normed field. So when I say ``vector space'', I mean a real one.

\begin{definition}
    A \textbf{norm} on a vector space $E$ is a map $\norm{-}:E\rightarrow \bR_{\geq0}$ such that for all $x,y\in E,\lambda \in \bR$: \[
    \norm{x}=0 \iff x = 0 \qquad \norm{\lambda x}=\abs{\lambda}\norm{x} \qquad \norm{x+y} \leq \norm{x} + \norm{y}
    \] called \textbf{definiteness}, \textbf{homogenaity}, and the \textbf{triangle inequality} respectively.
\end{definition}

Common examples are $\norm{-}_1$, $\norm{-}_2$, and $\norm{-}_\infty$ on $\bR^n$.

\begin{definition}
    For a compact $K\subseteq \bR^n$ the \textbf{uniform norm} on the set of continuous functions $\cC(K,\bR)$ is given by: \[
    \norm{f}_\infty := \sup_{x\in K}|f(x)| = \max_{x\in K}|f(x)|.
    \]
\end{definition}

Letting $K$ be $n$ discrete points recovers $\norm{-}_\infty$ on $\bR^n$, how can we recover the other defintions like this?

\begin{definition}
    A \textbf{Banach space} is a complete normed vector space.
\end{definition}

We will show in the finite case that all normed vector spaces are complete. Note that the on the space $\cC([0,1],\bR)$ the norm: \[
    \norm{f} := \int_0^1f(x)dx
\] is not complete, many progressive continuous approximations to non-conintuous shapes are an example. The change is normally increasingly localised and extreme, the uniform norm will properly detect this.

\begin{definition}
    Two norms, $\norm{-},\norm{-}'$ on a vector space $E$ are \textbf{equivalent} if there exists $C_1,C_2\geq 0$ such that for all $x\in E$: \[
        C_1\norm{x}' \leq \norm{x} \leq C_2\norm{x}'
    \]
\end{definition}
This is obviously an equivalence relation. Equivalent norms induce homeomorphic topologies by $\id$, as:\[
    B'_{r/C_2}(x)\subseteq B_r(x)\subseteq B'_{r/C_1}(x).
\]

\begin{proposition}
    All finite dimensional normed vector spaces are equivalent.\begin{proof}
        Choose any norm on a $d$ dimensional vector space $E$, and let $\{e_1,\ldots,e_d\}$ be a basis for $E$: \[
        \norm{x} = \norm{\sum x_ie_i} \leq \sum\norm{x_ie_i} = \sum \abs{x_i}\norm{e_i}\leq \max_{1\leq i\leq d}\norm{e_i}\sum\abs{x_i} = \max_{1\leq i\leq d}\norm{e_i}\norm{x}_1.
        \] Now, let $K$ be the unit sphere for $\norm{-}_1$. This is closed and bounded so, by the Heine-Borel theorem, compact. The previous step shows $x\mapsto\norm{x}$ is continuous wrt $\norm{-}_1$, so it will attain its infinum on $K$. By definiteness, this minimum, $m$, will be positive. And for all $x\in E$: $\norm{x}=\norm{x}_1 \norm{x/\norm{x}_1}\geq m\norm{x}_1$.
    \end{proof}
\end{proposition}

From now onwards, let $\Omega$ be an open subset of $\bR^d$ such that $\overline{\Omega}$ is compact and $\partial\Omega$ is smooth. The particulars of these restrictions isn't important.

\begin{definition}
    The \textbf{uniform norm} for $f\in \cB(\Omega,\bR)$ is given by $\norm{f}_\infty := \sup_{x\in \Omega}\abs{f(x)}$.
\end{definition}

Think of $C_b^0(\Omega,\bR)$ as a subspace of $\cB(\Omega,\bR)$.

\begin{proposition}
    $C_b^0(\Omega,\bR)$ is closed (contains all limit points) w.r.t.~the uniform norm.
    \begin{proof}
        Let $(f_n)\rightarrow f$, and fix some $\varepsilon > 0$. By assumption, there exists some $N\geq 0$ such that $\norm{f_N-f}_\infty < \varepsilon$. As $f_N$ is continuous, there will exist some $B_\delta(x)\subseteq \Omega$ such that for all $y\in B_\delta(x)$ we have $|f_N(x)-f_N(y)| < \varepsilon$. So we can write: \begin{align*}
            \abs{f(x)-f(y)} &= \abs{(f(x)-f_N(x)) - (f(y)-f_N(y)) + (f_N(x)-f_N(y))}\\
            &\leq \abs{(f(x)-f_N(x))} + \abs{(f(y)-f_N(y))} + \abs{(f_N(x)-f_N(y))} < 3\varepsilon. \qedhere
        \end{align*}
    \end{proof}
\end{proposition}

\begin{proposition}
    $C_b^0(\Omega,\bR)$ is complete w.r.t.~the uniform norm.
    \begin{proof}
        Let $(f_n)$ be Cauchy w.r.t~$\norm{-}_\infty$, for any $x\in \Omega$, $(f_n(x))$ will be a Cauchy sequence in $\bR$ so converges to a limit we will call $f(x)$. It now suffices to show $(f_n)\rightarrow f$ w.r.t.~$\norm{-}_\infty$ and appeal to the previous proposition. Fix $\varepsilon > 0$, by assumption there exists $N$ such that $\forall n,m\geq N$, $\norm{f_n-f_m}_\infty < \varepsilon$. Consider for all $x\in \Omega$: \[
        \abs{f_n(x)-f(x)} = \lim_{m\rightarrow\infty}\abs{f_n(x)-f_m(x)} \leq \varepsilon.
        \] As each $\abs{f_n(x)-f(x)}\leq$ is  $\varepsilon$, so will $\norm{f_n-f}_\infty$, the supremum.
    \end{proof}
\end{proposition}

\begin{definition}
    Let $\cC^k(\Omega,\bR)$ be the set of $k$-times bounded continuously differentiably functions, equipped with the norm: \[
    \norm{f}_{\cC^k} := \sum_{\abs{\alpha}\leq k} \norm{\partial^\alpha f}_\infty \qquad \alpha \in \bN^d
    \]
\end{definition}

The simplest example is with $d=k=1$ where $\norm{f}_{\cC^1} = \norm{f}_\infty + \norm{f'}_\infty$.

\begin{proposition}
    For $\Omega\subset \bR$, $\cC^1(\Omega,\bR)$ with the $\norm{-}_{\cC^1}$ norm is complete.
    \begin{proof}
        Let $(f_n)$ be Cauchy w.r.t.~$\norm{-}_{\cC^1}$, then both $(f_n)$ and $(f'_n)$ must be Cauchy w.r.t.~$\norm{-}_\infty$. So, by the previous proposition, there exists continuous, uniform limits $(f_n)\rightarrow f$, $(f'_n)\rightarrow g$. Thus: \[
        \norm{f_n-f}_{\cC^1} = \norm{f_n-f}_\infty + \norm{f'_n-g}_\infty \rightarrow 0.\]
        How do we know $f'=g$ or even $f\in\cC^1$? By FTC and the exchange of uniform limits with Riemann integrals, we have: \[
        f(x)-f(y) = \lim_{n\rightarrow\infty}f_n(x)-f_n(y) = \lim_{n\rightarrow\infty}\int_x^yf_n' = \int_x^yg
        \] for all $x,y\in\Omega$, and all $n$.
    \end{proof}
\end{proposition}

TODO: H\"older.

\subsection{Lebesgue spaces of functions}

We will keep the same fixed $\Omega$, and continue to be interested in functions $f:\Omega\rightarrow\bR$; and, unless stated otherwise, by: \[
\int_\Omega f(x)dx
\] I mean the integral w.r.t.~the Lebesgue measure. We can generalise to functions from $\Omega$ to some arbitrary measure space, but won't in this course.

\begin{definition}
    For some real $p\geq 1$, the \textbf{$L^p$-norm} on the set of measurable functions $f:\Omega\rightarrow \bR$ is: \[
    \norm{f}_{L^p} := \abs{\int_\Omega \abs{f}^pdx}^{\frac{1}{p}}
    \]
    and for $p=\infty$: \[
    \norm{f}_{L^\infty} := \essup_\Omega f = \inf\{M\in\bR, \ \abs{f}\leq M \text{ a.e.}\}
    \]
    Unfortunately, if $f=g$ a.e.~but $f\neq g$ our norm fails to be definite. For this we consider $L^p(\Omega,\bR)$ the set of such measurable functions with finite $L^p$ modulo equality a.e.,~making our norm definite.
\end{definition}

Most of the discussion around $L^p$ spaces has a counterpart in $l^p$ spaces, either by adapting the proof directly, or consider the collection of sequences as a discrete measure space. Such discussions probably won't be included.

To show the $L^p$ norm is in fact a norm, we know just have to show homogenaity, which is obvious, and the triangle inequality, which will involve some work.

\begin{lemma}[Young's inequality]
    Let $p,q,r\in\bR$ be such that \[
    \frac{1}{p} + \frac{1}{q} = \frac{1}{r}
    \] then for all $x,y>0$ we have \[
        (xy)^r \leq \frac{r}{p}q^p + \frac{r}{q}y^q
    \]\begin{proof}
        On problem sheet 2.
    \end{proof}
\end{lemma}

\begin{proposition}[H\"older's inequality]
    Given $p,q,r\in[1,\infty]$ such that \[
    \frac{1}{p} + \frac{1}{q} = \frac{1}{r}
    \] if $f,g\in L^p,L^q$ respectively, then $fg\in L^r$ and $\norm{fg}_{L^r}\leq\norm{f}_{L^p}\norm{g}_{L^q}$.\begin{proof}
        If $p$ or $q=\infty$ this is easy, so assume $p,q<\infty$. By homogenaity we may assume \[\norm{f}_{L^p} = \norm{g}_{L^q} = 1\] and by Young's inequality we see \[
            \norm{fg}_{L^r}^r = \int\abs{f}^r\abs{g}^rdx \leq \int\frac{r}{p}\abs{f}^p + \frac{r}{q}\abs{g}^qdx = \frac{r}{p}\int\abs{f}^pdx + \frac{r}{q}\int\abs{g}^qdx = \frac{r}{p} + \frac{r}{q} = 1.\qedhere
        \]
    \end{proof}
\end{proposition}

\begin{proposition}[Minkowski's inequality]
    If $f,g\in L^p$, then $\norm{f + g}_{L^p} \leq \norm{f}_{L^p} + \norm{g}_{L^p}$.\begin{proof}
        If $p=\infty$ this is easy, so assume $p=\infty$. Pointiwse, we know: \[
        \abs{f + g}^p \leq \abs{f + g}^{p-1}(\abs{f} + \abs{g})
        \] now integrate to get: \[
        \norm{f+g}_{L^p}^p=\int\abs{f + g}^pdx \leq \int\abs{f + g}^{p-1}(\abs{f} + \abs{g})dx = \int\abs{f + g}^{p-1}\abs{f}dx+\int\abs{f + g}^{p-1}\abs{g}dx
        \] applying H\"older's inequality on $\frac{p-1}{p} + \frac{1}{p} = 1$ gives this\[
        \leq \abs{\int\abs{f+g}^pdx}^{1-\frac{1}{p}}(\norm{f}_{L^p}+\norm{g}_{L^p}) = \frac{\norm{f+g}_{L^p}^p}{\norm{f+g}_{L^p}}(\norm{f}_{L^p}+\norm{g}_{L^p})\] multiplying out now completes the proof.
    \end{proof}
\end{proposition}

\begin{theorem}
    $L^p(\Omega,\bR)$ is complete w.r.t.~the $L^p$ norm.
    \begin{proof}
        Let $(f_n)$ be a Cauchy sequence in $L^p$, as subsequences of Cauchy sequences are also Cauchy sequences, wlog we can have $\norm{f_{n+1}-f_n}_{L^p}< 2^{-n}$. We want to define: \[
            f(x) = f_0(x) + \sum_{n=0}^\infty(f_{n+1}(x)-f(x))
        \] but we don't yet know if the sum is well defined, so instead consider \[
            g(x) = \abs{f_0(x)} + \sum_{n=0}^\infty\abs{f_{n+1}(x)-f(x)}
        \] which does exist, and call the $N$th trucation $g_N$. By Minkowski's inequality: \[
            \norm{g_N}_{L^p} \leq \norm{f}_{L^p} + \sum_{n=0}^N\norm{f_{n+1}-f_n}_{L^p} \leq C + \sum_{n=0}^N2^{-n} \leq C + 2.
        \] $(g_N)$ is a monotone increasing sequence converging to $g$ all $\geq 0$, the same holds for $(g_N^p)$ and $g^p$, so by the monotone convergence theorem: \[
            \norm{g}_{L^p}^p = \int\lim_{N\rightarrow\infty}\abs{g_N}^pdx = \lim_{N\rightarrow\infty}\norm{g_N}_{L^p}^p\leq(C+2)^p
        \] so $g$ is finite a.e.~and thus $f$ is defined a.e.

        $f$ is certainly in $L^p$ as $\abs{f}\leq g\in L^p$.

        Now fix an $\varepsilon>0$ and consider \[
        \abs{f-f_n} = \abs{f-f_0 - \sum_{k=0}^\infty\abs{f_{k+1}-f_k}}\leq \abs{f}+\abs{f_0} + \sum\cdots < \abs{f} + \abs{f_0} + g
        \]so \[
        \abs{f-f_n}^p \leq 3^p(\abs{f}^p+\abs{f_0}^0 + \abs{g}^p)
        \] the RHS is finite and doesn't depend on $n$ so by the domianted convergence theorem $(f-f_n)\rightarrow 0$ a.e.~thus $\norm{f-f_n}_{L^p}\rightarrow 0$.
    \end{proof}
\end{theorem}

\begin{proposition}
    Continuous functions are dense in $L^p$.
    \begin{proof}[Idea]
        We can reduce the situation to only deal with characterisitc functions of measurable sets $\mathbbm{1}_A$, by the regularity of the Lebesgue measure we can always find a closed $F$ and open $U$ such that $\mu(U\setminus F) < \varepsilon$ and $F\subset A \subset U$. Now we can observe: \[
        f(t) = \frac{\dist(t,\Omega\setminus U)}{\dist(t,\Omega\setminus U) + \dist(t,F)}
        \] has suppose on $U$, and is $1$ on $F$, and thus $\norm{f-\mathbbm{1}_A}_{L^p}<\varepsilon$.
    \end{proof}
    TODO: i have no idea what the fuck this is saying, i should learn what it means.
\end{proposition}

\subsection{Convolutions and mollifiers}

\begin{definition}
    If $f\in L^1(\bR^d)$ and $\phi\in\cC^0_c(\bR^d)$, then $f*\phi:\bR^d\rightarrow \bR$ is given by \[
    (f*\phi)(x) = \int_{\bR^d}f(x-y)\phi(y)dy
    \]
\end{definition}

Some obvious first properties of the convolution are that $\Supp(f*\phi)\subset \Supp(f) + \Supp(\phi)$, and $f*\phi = \phi*f$ if both are in $\cC^0_c$. To go further we need this following result we will not prove.

\begin{proposition}[Generalised Minkowski inequality]
    For \textit{suitable} $F$, \[
    \norm{\int F(x,y)dy}_{L^p(x)} \leq \int{\norm{F(x,y)}_{L^p(x)}dy}
    \] Instead of interchanging $L^p$-norms with sums, we interchange $L^p$-norms with integrals.
\end{proposition}

\begin{lemma}
    If $f\in L^p$ and $\phi\in\cC_c^0$ \[
    \norm{\phi*f}_{L^p} \leq \norm{\phi}_{L^1}\norm{f}_{L^p}
    \]\begin{proof}
        Applying the generalised Minkowsky inequality diretly to the definition of the convolution \[
        \norm{\int f(x-y)\phi(y)dy}_{L^p(x)} \leq \int\norm{f(x-y)\phi(y)}_{L^p(x)}dy = \int\abs{\phi(y)}\norm{f(x-y)}_{L^p(x)}dy
        \] TODO: there is something missing here, idfk what
    \end{proof}
\end{lemma}

\begin{proposition}
    If $f\in L^p$ and $\phi\in \cC_c^k$, then $f*\phi\in\cC^k$ and for all $\alpha\in\bN^d$, we have $\partial^\alpha(f*\phi) = f*\partial^\alpha\phi$.
    \begin{proof}
        Once again, only consider the case $k=d=1$, so $\partial^\alpha = \partial_x$. Furthermore, it suffices to deal with $f\in L^1$ as convolution is a local process and $L^p$ functions are locally $L^1$. \[
        \frac{(f*\phi)(x+h)-(f*\phi)(x)}{h} = \int f(y)\frac{\phi(x+h-y)-\phi(x-y)}{h}dy
        \] and the assosiation of $y$ to the RHS of the integrand, for fixed $x$, is uniformly bounded and converges pointwise to $\phi'(x-y)$, the claim now follows from the dominated convergence theorem.
    \end{proof}
\end{proposition}

The \textbf{mollifiers} $\varphi_n$ are defined as $\varphi_n(x) = n^d\varphi(nx)$ wherer $\varphi\in\cC^\infty_c$ and $\int\varphi=1$. These looks like bumb functions with increasing height and decreasing width as $n\rightarrow\infty$.

\begin{theorem}
    Given a sequence of mollifiers $(\varphi_n)$
    \begin{enumerate}
        \item If $f\in\cC_c^0$, then $f*\varphi_n\rightarrow f$ uniformly.
        \item If $f\in L^p$, then $f*\varphi_n\rightarrow f$ in $L^p$.
    \end{enumerate}
    \begin{proof}
        
    \end{proof}
\end{theorem}


\end{document}