\documentclass{article}
\usepackage{../../style/header}

\begin{document}

\title{Galois Theory}
\author{Lectured by Alessio Corti \\
Scribed by Yu Coughlin}
\date{Autumn 2025}

\maketitle

\tableofcontents

\section{Motivation}

\subsection{Solving harder polynomials}

Corti's way of teaching Galois theory is a charming combination of categorifying the main theory, while involving lots of computational methods. It is probably wise to try and approach his spartan expectation of your command of algebraic manipulation.

\begin{proposition}
    Quadratic equations over a field $k$ can be \textit{solved by radicals}.\footnote{This has a precise definition we \textit{might} talk about later on in this course, but you have an intuitive idea of what it really means.}
    \begin{proof}
        Have $aX^2+bX+c\in k[X]$, and execute the following chain of algebra: \[
            aX^2 + bX + c = a\pr{X-\frac{b}{2a}}^2-\frac{b^2}{2a} + c = 0 \iff X -\frac{b}{2a} = \pm\frac{\sqrt{b^2-4ac}}{2a}\qedhere
        \]
    \end{proof}
\end{proposition}

\begin{proposition}
    Cubic equations over a field $k$ can be solved by radicals.
    \begin{proof}
        Have $aX^3+bX^2+cX+d\in k[X]$, we first want to get rid of the $X^2$ by ``completing the cube''\[
        aX^3+bX^2+cX+d = a\pr{X-\frac{b}{3a}}^3 - \pr{\frac{b^2X}{3a} + \frac{b^3}{27a^3}} + cX + d 
        \] So if we divide through by $a$ and substitute $Y=X-b/3a$, we get a \textbf{depressed cubic} of the form $Y^3+pY+q=0$. I'll then use the substitution $Y = u - p/3u$:\footnote{I know this to be called ``Vieta's substitution''.} \[
        \pr{u-\frac{p}{3u}}^3 + p\pr{u-\frac{p}{3u}} + q = u^3 - pu + \frac{p^2}{3u} - \frac{p^3}{27u^3} + pu - \frac{p^2}{3u} + q = u^3 + q - \frac{p^3}{27u^3}
        \] This is now a quadratic in $u^3$, so we can use the previous method and find $6$ values for $u$, only $3$ of these will be distinct, we can then substitue back into $Y$ and then $X$.
    \end{proof}
\end{proposition}

\begin{proposition}
    Quartic equations over a field $k$ can be solved by radicals.
    \begin{proof}
        Let's just let this one be monic $f(X) = X^4+mX^3 + pX^2 + qX + r \in k[X]$ and do some algebra: \begin{align*}
            \pr{X^2+\frac{m}{2}X + \frac{\lambda}{2}}^2 = X^4 + mX^3 + \pr{\frac{m^2}{4} + \lambda}X^2 + \frac{m\lambda}{2}X + \frac{\lambda^2}{4}\\
            \implies \pr{X^2+\frac{m}{2}X + \frac{\lambda}{2}}^2 - \br{\pr{\frac{m^2}{4}+\lambda - p}X^2 + \pr{\frac{m\lambda}{2}-q}X + \pr{\frac{\lambda^2}{4} - r}} = f
        \end{align*}
        I want to make the right quadratic a perfect square, so I need the discriminant to be zero: \[
        \pr{\frac{m\lambda}{2}-q}^2 - 4\pr{\frac{m^2}{4}+\lambda - p}\pr{\frac{\lambda^2}{4} - r} = 0
        \] This is a cubic in $\lambda$, so we can use the previous method to solve it. Substituting, probably the nicest, solution for $\lambda$ back in lets us write our expression for $f$ as the difference of two squares, factoring our quartic as the product of two quadratics which we can solve with the first method.\footnote{I know this as ``Ferrari's method''.}
    \end{proof}
\end{proposition}

I have not really proved that the last two methods work, do my $\lambda$ and $u$ always exists, do the multiple roots all collapse nicely, how did people come up with this? I'm content not knowing any of these details and just using these algorithms, because they do in fact always work.

\subsection{Field stuff}

In this section, fix a field $\bQ\subset K \subset \bC$. For some $\alpha\in \bC$ we will use the notation: \[
K(\alpha) := \left\{
    \frac{P(\alpha)}{Q(\alpha)} \in \bC \ \middle| \ P,Q\in K[X], \ Q(\alpha)\neq 0
\right\}.
\]
$K(\alpha_1,\ldots,\alpha_n)$ is defined recursively.

\begin{definition}
    Such an $\alpha\in \bC$ is \textbf{algebraic over} $K$ is there is some nonzero polynomial $P\in K[x]$ such that $P(\alpha) = 0$.
\end{definition}

Consider $\bQ(\sqrt{2})$, hours of highschool mathematics tells us we don't need to consider the entire large set of quotients, just: \[
\bQ(\sqrt{2}) + \{a + b\sqrt{2} \mid a,b\in\bQ\}.
\]
If we choose something \textbf{transcendental} (non-algebraic) like $\bQ(\pi)$, then we must use the full quotient definition, and in this case we have $\bQ(\pi)\cong \bQ(X)$ the field of fractions of $\bQ[X]$.

\begin{definition}
    For some $f\in K[x]$ with distinct complex roots $a_1,\ldots,a_n\in\bC$, the \textbf{splitting field} of $f$ is $L=K(\alpha_1,\ldots,\alpha_n)$.
\end{definition}

Let $K=\bQ$ and $f=x^3-2$. The roots of $f$ in $\bC$ are $\sqrt[3]{2}$, $\omega \sqrt[3]{2}$, and $\omega^2\sqrt[3]{2}$, where $\omega$ is $(1-i\sqrt{3})/2$. So the splitting field is $L=\bQ(\sqrt[3]{2},\omega\sqrt[3]{2},\omega^2\sqrt[3]{2})$ which can be simplified to just $\bQ(\sqrt[3]{2},i\sqrt{3})$. What are the intermediate fields between $\bQ$ and $L$? 
\vspace{-10pt}
\[
\begin{tikzcd}[column sep=small]
& &[-50pt] L = \bQ(\sqrt[3]{2},i\sqrt{3})\\
\bQ(\sqrt[3]{2}) \arrow[urr] 
& \bQ(\omega\sqrt[3]{2}) \arrow[ur] 
& &[-50pt] \bQ(\omega^2\sqrt[3]{2}) \arrow[ul]\\[-20pt]
& & & & \bQ(i\sqrt{3})\arrow[uull, end anchor={[xshift=-4ex]south east}]\\
& & \bQ \arrow[uull] \arrow[uul] \arrow[uur] \arrow[urr]
\end{tikzcd}
\] Lots of fields you might guess, like $\bQ(i\sqrt{3}\sqrt[2]{3})$ and $\bQ(\sqrt[2]{3} + i\sqrt{3})$ happen to already be in this diagram. But we cannot yet prove this is everything. The length of the arrows is a clue, they relate to the dimension as $\bQ$-vector spaces and subgroups in the Galois correspondence.

\begin{theorem}[Fundamental theorem of Galois theory]
    The \textbf{Galois group} of a field extension $K\subset L$ is \[
        G = \Gal(L/K) = \left\{\varphi:L \xrightarrow{\sim} L \ \middle| \ \varphi\vert_K = \id_K\right\}
    \] and the eponymous Galois correspondence:\[
    \begin{tikzcd}[row sep = tiny]
        \phantom{..................}\{K\subset F \subset L\} \arrow[r, leftrightarrow, "\sim"] & 
        \{H\leq G\}\phantom{.....................................}\\
        \phantom{.....................................}F \arrow[r, maps to] & 
        F^\dagger := \{g\in G \mid g\vert_F = \id_F\} = G_F\\
        H^* := \{\alpha \in L \mid H\alpha = \alpha\} & 
        H \arrow[l, maps to] \phantom{.................................................}
    \end{tikzcd}.
    \]
\end{theorem}

If one knows the Galois group is a supgroup of the permutation of all the roots, then for the case $L=\bQ(\sqrt[3]{2},i\sqrt{3})$, there is seemingly no way to distinguish the roots, so we expect $G=S_3$, which is luckily true.

Fields are complicated and hard, there are two operations that ``cavort'' via a weird distributivity law, and proving the classification of subfields of $\bQ(\sqrt[3]{2},i\sqrt{3})$ already feels pretty impossible. Galois theory allows us to move information from the easy theory of finite groups into the world of field extensions.

\begin{proposition}
    $G$ is a subgroup of $\fS_n$, where $n$ is the degree of $f$.
    \begin{proof}
        We want to exhibit any $\sigma\in G$ as a permutation of the roots of $f$.
        Any such $\sigma$ is already an automorphism, so we just need to check it sends roots to roots. If $f(\lambda)=0$, for some $\lambda \in L$, then we can rearrange $f(\sigma(\lambda)) = \sigma(f(\lambda)) = 0$.

        So we get a homomorphism $\rho : G \rightarrow \fS_n$. Any automorphism in the kernel must fix every root, and so all of $L$, thus is the identity. Therefore, $\rho$ is injective and we get $G\leq \fS_n$.
    \end{proof}
\end{proposition}

I claim this will be enough Galois theory to prove that the above diagram for $x^3-2$ is everything! Because in this case, $G$ is all of $\fS_3$. To prove this I just have to construct enough elements. In general, constructing elements of Galois groups is very hard. We've already accepted general fields are very baroque, general automorphisms between them will surely be even less manageable. There are some easy cases.

\begin{proposition}
    For some $a$ not a perfect square in $K$, the Galois group of the field extension $K\subset K(\sqrt{a})$ will be $C_2$.
    \begin{proof}
        The automomophism $a+b\sqrt{a}\mapsto a-b\sqrt{a}$ is certainly nontrivial, this is all of $\fS_2$.
    \end{proof}
\end{proposition}

Now consider the quadratic extension in the diagram above: $K=\bQ(\sqrt[3]{2})\subset L = K(i\sqrt{3})$, the generator of $\Gal(L/K)$ is $\tau$ such that $\tau(i\sqrt{3}) = -i\sqrt{3}$, this is certainly part of all of $G$ as if it fixes $\bQ(\sqrt[3]{2})$ it definitely fixes $\bQ$. If we now explicitly calculate how this permutes the roots of $f$, we see $\tau = (2 \ 3)$.

Let's now do the same with $\bQ(\omega\sqrt[3]{2})$, we'll see the generator of this Galois group is $(1 \ 3)$. I don't actually need to compute how this $\tau$ acts on the roots, certainly it fixes $\omega\sqrt[3]{2}$, and certainly it is non-trivial, so it can only be $(1 \ 3)$.

These two roots generate all of $\fS_3$, so we have naively computed the galois group of $\bQ(\sqrt[3]{2},i\sqrt{3})$. BUT, we still haven't show that there are no other fields missing from our diagram, we must still simply believe in the full Galois correspondence to continue.

\section{Actual Stuff}

All field homomorphisms are injective\footnote{If $\phi:K\rightarrow L$ is a  field homomorphism then $\ker(\phi)\unlhd K$ is either $0$, making $\phi$ injective, or $K$, a contradicion as $\phi(1)=0$.}
so I'll write $\Emb(K,L)$\footnote{$\Emb$ stands for embedding.} for $\Hom(K,L)$ to remind us. Typically there is a fixed field $k$ in the background, and all fields $L$ come with a fixed embedding $i:k\hookrightarrow K$ which we normally supress.\footnote{
    We normally think of $k=\bQ$, in which case there is only ever one embedding of $\bQ$ into any algebraic extension.
} If $K,L$ are two such fields, we should only consider embeddings that fix $k$: \[
    \Emb_k(K,L) := \left\{f:K\rightarrow L \ \middle| \ f\vert_k = \id_k\right\}\footnote{more specifically, $f$ should commute with the two embeddings.}
\] In particular, $K$ will be a $k$-vector space. We can gain a lot of information by thinking about these field extensions in term of representations over our Galois groups, we won't talk about that here though.

\begin{definition}
    A field extension $K\subset L$ is \textbf{finite} if $L$ is finite as a $K$-vector space. The \textbf{degree} $[L:K]$ of the extension is $\dim_KL$.
\end{definition}

\begin{proposition}[Tower Law]
    If $K\subset L \subset M$ is a tower of field extensions, then \[[M:K]=[M:L][L:K].\]\begin{proof}
        TODO: finite case is just counting bases, this will suffice.
    \end{proof}
\end{proposition}

\subsection{Minimal polynomials}

\begin{definition}
    Given $K\subset L$, for all $a\in L$, the \textbf{minimal polynomial} of $a$ in $K$ is the monic generator of the kernel of the ring homomorphism: \[
        \ev_a : K[X] \rightarrow L \qquad \phi(x) \mapsto \phi(a)
    \] Somewhere in a previous algebra course, we have proved $K[X]$ is an Euclidean domain, so a PID.
\end{definition}

The first isomorphism theorem for rings now tells us: \[
    K[X] / (f) \cong \im(\ev_a) \subset L
\] so $\im(\ev_a)$ is an integral domain, and thus $f$ is prime so irreducible, and so $(f)$ is maximal and $\im(\ev_a)$ is a field. Thus $K[a]$, the $K$-algebra generated by $a$, is the same as $K(a)$ the field generated by $a$ as defined earlier.\footnote{
    This is somehow not very constructive, our proofs that $K[X]$ is a PID normally goes by the ACC, we have no algorithm for computing $a^{-1}$ in $K[a]$.
}

Conversely, if $f\in K[X]$ is monic and irreducible, and $L$ is the field $K[X]/(f)$, then $K\subset L$ is a finite field extension. If we set $a:=[X] \in L$, then $L=K[a]=K(a)$, and $f(x)$ is the minimal polynomial of $a$. The degree of such an extension $[K:K(a)]$ is equal to the degree of the minimal polynomial.

Finally, for a field extension $K\subset L$ and all $a \in L$ we will always have: \[
    \Emb_K(K(a),L) = \{\text{roots of $f$ in $L$}\}
\]

\subsection{Splitting fields}

The construction of a splitting field of a $K$-irreducible polynomial as $L=K[X]/(f)$ is still important. When $L\subset \bC$ we can adjoin sufficiently many roots of $f$ and have a nice constructive form, but we are still interested fields that are not part of $\bC$.

For example, consider $\bF_2$ and the irreducible polynomial $X^2+X+1\in\bF_2[X]$, we can form $L = \bF_2[X]/(X^2+X+1)$ which he know has degree $2$ of $\bF_2$ so has $4$ elements. Are there any other fields of $4$ elements? If one exists it is certianly of characteristic $2$ so contains $\bF_2$, any remaining element not contained in $\bF_2$ has a minimal polynomial of degree $2$ which must be irreducible, this can only be $X^2+X+1$ so there is only one such field of size $4$ up to isomorphism. We can call this $\bF_4$.\footnote{There is a certainly a Galois action that swaps the two roots of $X^2+X+1$, so while our field is defined only up to non-unique isomorphism. Is $\bC$ the same? can we distinguish $i$ from $-i$?} In general, we know the theory of finite fields to be somewhat well behaved and all formed in a similar fashion. What about infinite fields of finite characteristic? Consider \[
    K := \bF_2(t) = \Frac(\bF_2[t]) = \left\{ \frac{P(t)}{Q(t)} \ \middle| \ P(t),Q(t)\in \bF_2[t], \ Q(t)\not\equiv 0\right\}
\] and the polynomial $X^2-t\in K[X]$, which is irreducible. We can certianly prove this is irreducible by naively constructing a root and finding a contradiction. Or we can endeavor to learn some irreducibility conditions for general polynomials.

\begin{proposition}[Eisenstein criterion]
    If $f = X^n + a_1X^{n-1} + \cdots + a_n \in \bZ[X]$ and there exists some prime $p\in \bZ$ such that $p\mid a_i$ and $p^2 \nmid a_n$, then $f$ is irreducible.
\end{proposition}

\begin{corollary}
    By Gauss' lemma we know this implies $f$ is also irreducible over $\bQ[X]$.
\end{corollary}

But the Eisenstein criterion is ``totally rubbish'' and in practice will rarely work.\footnote{It is like if you drop your keys on a dark street, the mathematician goes to look for your keys on the other side of the street where there is a lamp. Sure, \textit{if} your keys happen to have reached the other end of the street, they will certainly be found. But this is unlikely, just how it is unlikely an irreducible polynomial satisfies Eisenstein.} We could now use the Gauss lemma to simplify our naive strategy as we only have to search $\bF_2[t][X]$, but in fact the analogue between $\bZ$ and $K[X]$ holds for the Eisenstein criterion, and by using $p=t\in\bF_2[t]$, we deduce $X^2-t$ is irreducible. We can even observe $K[x]/(X^2-t)\cong K$, which you should percieve as being very weird!

But now, in more generality:

\begin{definition}
    Let $K$ be a field and $f\in K[x]$ be irreducible. A \textbf{splitting field} of $f$ over $K$ is a field extension $K\subset L$ such that: \[
        f(x) = \prod_{i=1}^n(X-\lambda_i) \in L[X]
    \] splits completely in $L[X]$; and $L$ is generated by $S=\{\lambda \in L \mid f(\lambda) = 0\}$ over $K$.
\end{definition}
Splitting fields certainly exist, we can just progressively adjoin all roots of $f$ until $f$ splits completely. I claim, if: \[
\begin{tikzcd}[row sep = small, column sep = tiny]
    L_1 \arrow[dr, phantom, sloped, "\supset"] & & L_2 \arrow[dl, phantom, sloped, "\subset"]\\
    & K
\end{tikzcd}
\] are two splitting fields for $f$, then there exists some ismorphism $\phi\in\Emb_K(L_1,L_2)$.

\subsection{Motivation again}

Back to the tower law to prove that we have a full subfield lattice:\[
\begin{tikzcd}[column sep=small]
& &[-50pt] L = \bQ(\sqrt[3]{2},i\sqrt{3})\\
\bQ(\sqrt[2]{3}) \arrow[urr, "2" description] 
& \bQ(\omega\sqrt[2]{3}) \arrow[ur, "2" description] 
& &[-50pt] \bQ(\omega^2\sqrt[2]{3}) \arrow[ul, "2" description]\\[-20pt]
& & & & \bQ(i\sqrt{3})\arrow[uull, end anchor={[xshift=-4ex]south east}, "3" description]\\
& & \bQ \arrow[uull, "3" description] \arrow[uul, "3" description] \arrow[uur, "3" description] \arrow[urr, "2" description]
\end{tikzcd}
\]
TODO: I know the diagram is ugly, I don't care right now.

By the tower law, we now know $[L:\bQ]=6$, so any addition subfields $K$ must have $[K:\bQ]=2$ or $3$. Suppose $[K:\bQ]=2$, for $K$ to be distinct it must not contain $i\sqrt{3}$. So $X^2+3$ remains irreducible over $K$ and we can form the extension:\[
\begin{tikzcd}[column sep=small, row sep = small]
    & K(i\sqrt{3})\\
    K \arrow[ur, "2" description]& & \bQ(i\sqrt{3}) \arrow[ul, "2" description]\\
    & \bQ \arrow[ul, "2" description] \arrow[ur, "2" description]
\end{tikzcd}
\]
But the tower law now gives us $6 = [L:\bQ] = [L:K(i\sqrt{3})][K(i\sqrt{3}):\bQ] = 4[L:K(i\sqrt{3})]$ and $4\nmid 6$.

If instead $[K:\bQ]=3$, for $K$ to be distinct it must contain no roots of $X^3-2$, so we can form $K[X]/(X^3-2)$, a degree three extension, and $9\nmid 6$.

\subsection{Back to splitting fields}

\begin{lemma}
    Let $f$ be an irreducible polynomial in $k\subset K$, and $K(a) = K[X]/(f)$, then for any $K\subset L$:\[
        \Emb_K(K(a),L) \cong \{\text{roots of $f$ in $L$}\}
    \]\begin{proof}
        consider the map which sends $\varphi\in\Emb_k(K(a),L)$ to $\varphi(a)\in L$ we know $\varphi$ will commute with $f$ so $\varphi(a)$ is a root. Indeed, this map is injective as if we have some $\phi(a) = \phi'(a)$ in $L$, they must now agree on all of $K(a)$. If isntead I have a root $b\in L$ of $f$, then the map $\ev_b:K[X]\rightarrow L$ must have $f$ the generator of the kernel, so the isomorphism theorem gives us a map $K(a) = K[X]/(f) \rightarrow L$.
    \end{proof}
\end{lemma}

We have already seen this to be true! There are three embeddings of $\bQ(\sqrt[3]{2})$ into $\bQ(\sqrt[3]{2},i\sqrt{3})$ as either $\bQ(\sqrt[3]{2})$, $\bQ(\omega\sqrt[3]{2})$, or $\bQ(\omega^2\sqrt[3]{2})$.\footnote{``We rrrrrrrip $\bQ(\sqrt[3]{2})$ out of its environment, it is an animal embedded in the jungle of complex numbers''. It is just isomorphic to the abstract field $\bQ[X]/(X^3-2)$, which exists as three different copies in $\bC$ ``This abstract copy can be incarnated as a lion or a tiger or a zebra''. We are just measuring the wasted space of the slpitting field.}

\begin{lemma}
    For $K$ a field and $f\in K[X]$ and suppose there are two extensions $K\subset L_1$, which is generated by the roots of $f$, and $K\subset L_2$ where $f$ splits completely, then\[
    \Emb_K(L_1,L_2)\neq \emptyset
    \]\begin{proof}
        If $K=L_1$, then we are done, otherwise choose some $a\in L_1\setminus K$ s.t. $f(a)=0$ and let $g\in K[X]$ be the minimal polynomial of $a$, certainly $g\mid f$, and so $g$ splits completely in $L_2[X]$. There must be another root $b\in L_2$ such that $g(b)=0$, this corresponds to an embedding $K(a)\subset L_2$ by the previous lemma. We now just need to show \[
        \Emb_{K(a)}(L_1,L_2)\neq \emptyset
        \] but as $[L_1:K(a)] < [L_1 : K]$, by induction we are done.
    \end{proof}
\end{lemma}

\begin{corollary}
    For $K$ a field and $f\in K[X]$ any two splitting fields of $f$ are isomorphic over $k$.
\end{corollary}

\subsection{Normal extensions}

\begin{definition}
    $k\subset K$ is \textbf{normal} if for all $k\subset \Omega$ and $x_1,x_2\in\Emb_k(K,\Omega)$ there exists a $\varphi \in \Gal(K/k)$ such that $x_1=x_2\varphi$
\end{definition}

Intutively, the image of any $k$-embedding $K\subset \Omega$ is fully determined by the embedding $k\subset \Omega$. The shape of a normal extension around $k$ is fully determined.

If, instead, we fix our `big' field extensions $k\subset \Omega$, then $\Emb_k(-,\Omega)$ becomes a functor from algebraic $k$-extensions to $\textbf{FinSet}$, and we always get a faithful \textbf{right} action of $\Gal(K/k)$ on $\Emb_k(K,\Omega)$ by pre-composition. Now $k\subset K$ is normal iff this action will be transitive, equivalently iff $\Emb_k(K,\Omega)$ is a $\Gal(K/k)$-torsor.\footnote{Something that looks like $\Gal(K/k)$ without a choice of identity.}

\begin{proposition}
    If $f(x)\in k[X]$ and $k\subset K$ is the splitting field of $f$, then $k\subset K$ is normal.
    \begin{proof}
        Consider $K\subset\Omega$, we want to show for any $x:K\hookrightarrow \Omega$, $x(K)\subset K$. If $a\in K$ is a root of $f$, then $x(a)$ is also a root of $f$, so $x(a)\in K$, as $K$ is generated by roots of $f$ we must have $x(K)\subset K$.
    \end{proof}
\end{proposition}

The converse is true but we will prove this later. Splitting fields are fully determined by a polynomial, and somehow \textbf{all} of the interacions between the elements are controlled by this polynomial, so the `shape' of $K$ in any bigger extensions is already totally controlled by a split polynomial.

\subsection{Separable extensions}

\begin{definition}
    $k\subset K$ is \textbf{separable} if for all towers of subfields \[
    k \subset F_1 \subset F_2 \subset K
    \] there exists a big field $k\subset\Omega$ with at least two distince $K_1$-embeddings $x,y:K_2\rightarrow \Omega$.
    I'll give a slightly down to earth criterion, but without proof for the moment.
\end{definition}


If $k=\bF_2(t)\subset K = k(\sqrt{t})$, the polynomial $X^2-t\in k[X]$ is irreducible, and $K$ is certianly its splitting field, but it only has one root in $K$. So, there is only one element of $\Emb_{F_1}(F_2,\Omega$). What is an example in the positive?

\begin{definition}
    A polynomial of degree $n$ is \textbf{separable} if it has $n$ distinct roots in its splitting field.
\end{definition}

TODO: polynomial derivative and jacobian criterian, fucking snooze

If a polynomial $f$ is irreducible, when is it also separable, iff $D(f)=0$, in characteristic $0$ this never happens, but in characteristic $p>0$ this is true iff $f(X)=h(X^p)$ for some $h\in k[X]$

\begin{definition}
    A field $F$ of characteristic $p$ is perfect if all elements are perfect $p$th powers.
\end{definition}

For such a field, the \textbf{Frobenius} map $\Fr_p(a)=a^p$ is a field embedding. When $F=\bF_p$, $\Fr$ is in fact the identity, as \[
a^p\equiv a\mod{p}
\]

\begin{proposition}
    Finite fields are always perfect.
    \begin{proof}
        The Frobenius map is an injective endomorphism of a finite set, so is surjective and thus every element is realised as a perfect $p$th power.
    \end{proof}
\end{proposition}

\begin{proposition}
    In a perfect field every polynomial is separable.\begin{proof}
        $f(X)=h(X^p)=g(X)^p$, as all the coefficients are perfect $p$th powers.
    \end{proof}
\end{proposition}

So we only discover irreducible and inseparable polynomials in infinite fields of characterstic $p>0$.

\subsection{Axioms}

\begin{proposition}[A1]
    For all $k\subset K \subset L$, the set $\Emb_k(K,L)$ is finite, in fact $\abs{\Emb_k(K,L)}\leq [K : k]$.
\end{proposition}

\begin{proposition}[A2]
    Every morphism is injective, and every endomorphism is also surjective.
\end{proposition}

\begin{proposition}[A3]
    If $k\subset K_1,K_2\subset L$, then $K_1\cap K_2$ and $K_1K_2$ (the smallest subfield containing $K_1$ and $K_2$) are both fields between $k$ and $L$.\footnote{The interesection is visibly the pullback, but the product isn't just the pushout, in the notes this is called a \textbf{framed pushout}, my understanding is that there is no actual pushout, only when we specify how $K_1$ and $K_2$ are embedded into a bigger field $\Omega$ there is a pushout in $k\textbf{Gal}\downarrow\Omega$. What else do I name this category???}
\end{proposition}

\begin{proposition}
    Suppose $k\subset K,L$ then there exists some finite extension $k\subset\Omega$ containing them both: \[
    \begin{tikzcd}[row sep = small, column sep = small]
        & K \arrow[dr, phantom, sloped, "\subset"]\\
        k \arrow[ur, phantom, sloped, "\subset"] \arrow[dr, phantom, sloped, "\subset"] && \Omega\\
        & L \arrow[ur, phantom, sloped, "\subset"]
    \end{tikzcd}
    \] In general, $\Omega$ is not unique.
\end{proposition}

\begin{proposition}[A5]
    If $K$ is a field and $G$ is a finite group acting faithfully on $K$ as field automorphisms, then: \[
    K^G = \{\alpha\in K \mid \forall g\in G, \ g(\alpha) = \alpha \}
    \] is a field, and the map $G\hookrightarrow \Gal(K/K^G)$ is an isomorphism.
\end{proposition}

TODO: phrase this properly and actually define the specific galois category we work in.

\end{document}