\documentclass{article}
\usepackage{../../style/header}

\begin{document}

\title{Algebraic Curves}
\author{Lectured by Steven Sivek \\
Scribed by Yu Coughlin}
\date{Autumn 2025}

\maketitle

\tableofcontents

\subsection{Varieties}

Throughout, let $k$ be a field.

\begin{definition}
    For a finite collection of polynomials $f_1,\ldots,f_m\in k[x_1,\ldots,x_n]$ we define the \textbf{affine algebraic variety}\footnote{There are complexities to this deifnition like irreducibility, but this is an introductory course so I won't focus on that notation.} $\bV_k(f_1,\ldots,f_m)$ as the set of common zeroes of all the polynomials in $k^n$.
\end{definition}

So many basic objects of geometry are captured as affine algebraic varieties: finite sets of points $\bV_k(x_1-a_1,\ldots,x_n-a_n)$, conics $\bV_\bR(ax^2+bxy+cy^2+dx+ey+f)$, affine spaces $\bA_k^n := \bV_K(0)$, and many others. We will restrict our scope of study in this course.

\begin{definition}
    An \textbf{affine plane curve} over $k$ is the affine variety: \[
        C = \bV_k(P) := \{(x,y)\in k^2 \mid P(x,y) = 0\}
    \] where $P$ is a non-constant polynomial in two variables.
\end{definition}

There are some natural first questions we can ask about affine plane curves: 
\begin{itemize}
    \item How does this geometrically represent the factorisation of our plane curve?
    \item How can we compute the number of intersections of two plane curves?
    \item When to two polynomials give the same plane curve?
    \item Can we classify all plane curves over certain fields?
    \item What is the appropriate notion of two curves being equivalent?
\end{itemize}

We'll begin to develop some of the basic theory of algebraic geometry so that we can state these questions more precisely.

\begin{theorem}[Hilbert basis theorem]
    If $R$ is a Noetherian ring, then so is $R[X]$
    \begin{proof}
        This is in my notes for algebra 3.
    \end{proof}
\end{theorem}

\begin{corollary}
    All ideals $I\unlhd k[x_1,\ldots,x_n]$ are finitely generated.
\end{corollary}

So I can expand the definition of an affine variety.

\begin{definition}
    The \textbf{affine variety} associated to an ideal $I\unlhd k[x_1,\ldots,x_n]$ is:\[
        \bV(I) := \{(a_1,\ldots,a_n)\in k^n \mid f(a_1,\ldots,a_n)=0 \text{ for all } f\in I\}
    \]
\end{definition}

The Hilbert basis theorem tells us any such $I$ will be finitely generated, i.e.\[
    I = (f_1,f_2,\ldots,f_m)
\] so we should have

\begin{lemma}
    $\bV(I) = \bV(f_1,\ldots,f_m)$.
    \begin{proof}
        ($\subset$) Take $(a_1,\ldots,a_n)\in \bV(I)$, certianly each of the $f_i\in I$, so $(a_1,\ldots,a_n)$ vanishes an all $f_i$ and is thus in $\bV(f_1,\ldots,f_m)$.

        ($\supset$) Now suppose $(a_1,\ldots,a_n)\in\bV(f_1,\ldots,f_m)$, we know for any $f\in I$, there exists $g_1,\ldots,g_m$ such that \[
            f = g_1f_1 + g_2f_2 + \cdots g_mf_m
        \] as each $f_i$ is zero at $(a_1,\ldots,a_n)$ so is $f$.
    \end{proof}
\end{lemma}

\begin{lemma}
    The set of affine varieties is closed under finite unions and arbitrary intersections.\begin{proof}
        Let $V_1=\bV(f_1,\ldots,f_m)$ and $V_2=\bV(g_1,\ldots,g_l)$, then I claim \[
        V_1\cup V_2 = \bV(f_ig_j \mid \text{for all } 1\leq i\leq m, \ 1\leq j\leq l)
        \]
        ($\subset$) Certainly if $(a_1,\ldots,a_n)\in V_1\cup V_2$ then it will be zero for all $f_i$ or all $g_j$, so will always be zero on all products $f_ig_j$.

        ($\supset$) If, instead, $a=(a_1,\ldots,a_n)$ is zero for all $f_ig_j$ and, wlog, $a$ is nonzero for some $f_i$, then as $f_ig_1,g_ig_2,\ldots,f_ig_l$ all have $a$ as a zero, and $k[x_1,\ldots,x_n]$ is an integral domain, $a\in V_2$.

        Now suppose $\{\bV(I_\alpha)\}_{\alpha\in A}$ is a set of varieties indexed by some set $A$, let $I=\bigcup_{\alpha\in A}I_\alpha$, I claim: \[
        \bigcap_{\alpha\in A}\bV(I_\alpha) = \bV(I)
        \] Certianly if $a$ is in all $\bV(I_\alpha)$, then it is a zero of all polynomials in each $I_\alpha$, which is precisely the polynomials in $I$. If $a\in\bV(I)$, then by the Hilbert basis theorem $I$ is finitely generated by some polynomials $f_1,\ldots,f_m$ appearing in $I_{\alpha_1},\ldots I_{\alpha_m}$ respectively, we know these polynomials generate $I$, so in fact generate all ofe very $I_\alpha$, so $a$ is in each $\bV(I_\alpha)$. TODO:terrible proof.
    \end{proof}
\end{lemma}

\begin{theorem}[Fundamental theorem of algebra]
    $\bC$ is algebraically closed.
\end{theorem}

\begin{proposition}
    If $k$ is algebraically closed, then any affine plane curve over $k$ has infinitely many points.\begin{proof}
        
    \end{proof}
\end{proposition}

\subsection{Irreducible plane curves}

\end{document}