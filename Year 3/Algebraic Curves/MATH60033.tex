\documentclass{article}
\usepackage{../../style/header}

\begin{document}

\title{Algebraic Curves}
\author{Lectured by Steven Sivek \\
Scribed by Yu Coughlin}
\date{Autumn 2025}

\maketitle

\tableofcontents

\appendix

\section{Commutative algebra}

\subsection{Noetherian modules}

\begin{definition}
    Given a commutative ring $R$, an $R$-module $M$ is \textbf{Noetherian} if all $R$-submodules are finitely generated.
\end{definition}

\begin{proposition}
    $M$ is Noetherian iff every ascending chain of submodules stabilises.
    \begin{proof}
        ($\Rightarrow$) Let $N_0\subseteq N_1 \subseteq \cdots$ be the ascending chain, define $N = \bigcup_i N_i$ then $N$ is finitely generated by some $\{m_1,\ldots,m_r\}$ each first appearing in $N_{i_1},\ldots,N_{i_r}$, the chain stabilises at the maximum of these.

        ($\Leftarrow$) If any submodule is not finitely generated we may form the never stabilising chain $N_0 = \abr{n_0}$, $N_{i+1} = N_i + \abr{n_{i+1}}$ where each $n_{i+1}\notin N_i$, so all submodules must be finitely generated.
    \end{proof}
\end{proposition}

\begin{definition}
    A ring $R$ is Noetherian if it is Noetherian as an $R$-module.
\end{definition}

\begin{lemma}
    All submodules of Noetherian modules are Noetherian.
\end{lemma}

\begin{lemma}
    All quotients of Noetherian modules are Noetherian.
\end{lemma}

\begin{proposition}
    If $N\leq M$ are $R$-modules and both $N$ and $M/N$ are Noetherian, so is $M$.
\end{proposition}

\begin{corollary}
    Any finitely generated module over a Noetherian ring is Noetherian.
\end{corollary}

\subsection{Hilbert basis theorem}

The leading coefficient of a polynomials $f = a_0 + a_1X + \ldots + a_nX^n$ is $a_n\in R$.

\begin{lemma}
    Given $R$ is a Noetherian ring, if $I\unlhd R[X]$, and $J$ is the set of leading coefficients of polynomials in $I$, then $J$ is an ideal of $R$.
    \begin{proof}
        If $a$ is the leading coefficient of $f\in I$, then for any $r\in R$, $ra$ will be the leading coefficient of $rf\in I$. If $b$ is the leading coefficient of $g\in I$ and the degree of $f$ and $g$ are $n$ and $m$ respectively. Wlog, take $n\leq m$, then $a+b$ will be the leading coefficient of $X^{m-n}f + g \in I$ which has degree $m$.
    \end{proof}
\end{lemma}

As $R$ is Noetherian, $J$ must be finitely generated by some $\{a_1,\ldots,a_s\}$. For each $a_i$ there must be a polynomial $f_i$ with leading coefficient $a_i$, say this has degree $d_i$; and let $d$ be the maximum of these.

\begin{lemma}
    For any $f\in I$, there exist $p_1,\ldots,p_i\in R[X]$ such that $\deg(f-p_1f_1-\cdots -p_sf_s) < d$.
    \begin{proof}
        By induction on $\deg(f)$. For the base case, suppose $\deg(f) < d$, then set all $p_i\equiv 0$. 
        
        Now, suppose the claim holds for all polynomials of degree less than some $e$. Let $a$ be the leading coefficient of $f$, as $a\in J$ we can write $a=r_1a_1 + \cdots r_sa_s$ for some $r_i\in R$, now consider: \[
        f - r_1X^{e-d_1}f_1 - \cdots - r_sX^{e-d_s}f_s
        \] which must not be of degree $< e$. As each $r_iX^{e-d_i}\in R[X]$ we are done.
    \end{proof}
\end{lemma}

\begin{theorem}[Hilbert basis theorem]
    If $R$ is a Noetherian ring, then so is $R[X]$.
    \begin{proof}
        Let $R[X]_{\leq d}$ be the $R$-submodule of $R[X]$ generated by $\{1,X,\ldots,X^d\}$ this is finitely generated so Noetherian. $I\cap R[X]_{\leq d}$ is a submodule so also finitely generated, let $\{g_1,\ldots,g_k\}$ be the generating set. The previous lemma now shows $I$ is generated by $\{g_1,\ldots,g_k,f_1,\ldots,f_s\}$.
    \end{proof}
\end{theorem}

\subsection{Nullstellensatz}

\subsection{Polynomial rings over UFDs are UFDs}

\end{document}