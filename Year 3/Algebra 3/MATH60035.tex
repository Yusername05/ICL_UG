\documentclass{article}
\usepackage{../../style/header}

\begin{document}

\title{Algebra 3}
\author{Lectured by Alession Corti \\
Scribed by Yu Coughlin}
\date{Autumn 2025}

\maketitle

\tableofcontents

\section{Rings and Modules}

\subsection{Introduction}
You hopefully already know the definition of a ring, so here are some examples:
\begin{itemize}
    \item $\bZ$, the ring of integers;
    \item any field like $\bF_{p^n}$, $\bQ$, $\bR$, $\bC$, and et cetera;
    \item for a given ring $R$, the ring of polynomials $R[x]$, when we let $R=k$ a field this is an Euclidean domain admitting many parallels to $\bZ$;
    \item $R[[x]]$, the ring of power series;
    \item $M_n(R)$ the ring of $n\times n$ matrices with entries in $R$, the first noncommutative example here;
    \item $R$-valued functions out of any set have a pointwise ring structure;
    \item For a group $G$ we can consider the group ring $R[G]$, which we formalise as the set of finitely supported maps $f:G \rightarrow R$, addition is pointwise, and multiplication is given by \[
    (fg)(x) := \sum_{uv=x}f(u)g(v)
    \]
    written as $R$-linear combinations of elements of $G$, this will be commutative iff G is abelian;
    \item we can define the quaternions as $\bH:= \bR[Q_8]$, where $Q_8$ is the quaternionic group $\{\pm 1,\pm i, \pm j,\pm k\}$;
    \item the first Weyl algebra is roughly ``the ring of polynomial valued differential operators'', defined as \[
        A_1 := \bC[x,\partial] / (\partial x - x \partial = 1)
    \] you should think of this acting of $f\in \bC[x]$, generated by the product rule: \[
    \partial xf - x\partial f = f\frac{d}{dx}xf - x\frac{d}{dx}f = f.
    \]
\end{itemize}
\section{Matrix Lie Groups}

\end{document}