\documentclass{article}
\usepackage{../../style/header}

\begin{document}

\title{Algebra 3}
\author{Lectured by Alession Corti \\
Scribed by Yu Coughlin}
\date{Autumn 2025}

\maketitle

\tableofcontents

\section{Rings and modules}

\subsection{Monoid rings}

\begin{definition}
    A \textbf{ring} is an abelian group $(R,+,0)$ which is also a monoid $(R,\cdot,1)$ such that $\cdot$ distributes over $+$. We may also require $0\neq 1$.
\end{definition}
Some classical examples are: \begin{itemize}
    \item $\bZ$, the ring of integers;
    \item any field like $\bF_{p^n}$, $\bQ$, $\bR$, or $\bC$;
    \item $M_n(R)$ the ring of $n\times n$ matrices with entries in $R$, the first noncommutative example here;
    \item $R$-valued functions out of any set have a pointwise ring structure;
    \item the first Weyl algebra is roughly ``the ring of polynomial valued differential operators'', defined as \[
        A_1 := \bC[x,\partial] / (\partial x - x \partial = 1)
    \] you should think of this acting of $f\in \bC[x]$, generated by the product rule: \[
    \partial xf - x\partial f = f\frac{d}{dx}xf - x\frac{d}{dx}f = f.
    \]
\end{itemize}

\begin{definition}
    For a monoid $P$ and a ring $R$ the \textbf{monoid ring} is the set of finite $R$-linear sums:\[
    \sum_{p\in P}r_p p.
    \] The addition and multiplication come from the independent inclusions $R,P\subset R[P]$ by $r\mapsto r1_p$ and $p \mapsto 1_Rp$ respectively.
\end{definition}

Obviously, every group is a monoid, but here are some other examples:\begin{itemize}
    \item $\bN$ is a semiring, so a monoid under addition;
    \item given a set $X$, both $\cup$ and $\cap$ make $\cP(X)$ a monoid with $\emptyset,X$ the respective identities;
    \item the set of endomorphisms of an object in a category is always a monoid;
    \item let $C\subseteq \bR^n$ be a convex cone, i.e. $\bR_{\geq 0}C = C$ and $C+C = C$, then $P=C\cap\bZ^n$ is a monoid under addition, if $C=\abr{(1,0),(-1,2)}$ is generated by poitns in $\bZ$ we may not necessarily have $P$ generated as a monoid by these same elements (in this case $(0,1)\in P$ but cannot be realised as a $\bZ$-linear combination of $(1,0)$ and $(-1,2)$).
\end{itemize}

Most of the canonical examples of rings missed out earlier was because they can be realised as monoid rings: \begin{itemize}
    \item $R[x_1,\ldots,x_n]$ the ring of polynomials in $n$ variables, is just the monoid ring $R[\bN^n]$;
    \item $R[x_1,x_1^{-1},\ldots,x_n,x_n^{-1}]$ the ring of Laurent polynomials in $n$ varialbes, is the monoid ring $R[\bZ^n]$;
    \item we will see that a lot of ring homomorphisms are induced by monoid homomorphisms they are rings over, a first example of this is the isomorphism $\bC[\bZ/n\bZ]\cong\bC[x]/(x^n-1)$;
    \item we can define the quaternions $\bH$ as a quotient of $\bR[Q_8]$, where $Q_8$ is the quaternionic group $\{\pm 1,\pm i, \pm j,\pm k\}$, by the ideal $(1 + (-1), i + (-i),j+(-j),k+(-k))$;
    \item $\bR[P]$ from earlier is a subring of $\bR[x,y]$ which, as $P$ isn't generated by $\{(1,0),(-1,2)\}$, is actually $\bR[x,y,y^2/x]$.
\end{itemize}

\begin{definition}
    For two rings $S,R$ a \textbf{ring homomorphism} is a function $f:R\rightarrow S$ such that $f$ is a homomorphism of the additive groups and multiplicative monoids.
\end{definition}

Constructing ring is in general very hard, it is a lot of data.

\begin{definition}
    If $R$ is a ring and $f:P\rightarrow Q$ is a monoid homomorphism then there is an \textbf{induced homomorphism} $f_* : R[P]\rightarrow R[Q]$ given by: \[
    \sum_{p\in P}r_p p \mapsto \sum_{p\in P} r_pf(p).
    \]
\end{definition}

This makes a lot of interesting ring homomorphisms, anything more interesting belongs to the land of algebraic geometry, we will not discuss that here.

The \textbf{image} and \textbf{kernel} of a ring homomorphism are inhereted from the additive group homomorphism.

\begin{definition}
    An additive subgroup $I\leq R$ is a \textbf{left ideal} if $RI \subseteq I$. Right ideals and two-sided ideals are defined obviously.
\end{definition}

The kernel of a ring homomrophism is certainly a two-sided ideal. Conversely, to any ideal $I$ there is a unique ring structure on $R/I$ that makes $\varphi:R\rightarrow R/I$ a ring homomorphism.

\begin{definition}
    A \textbf{left unit} in $R$ is an element $u\in R$ such that there exists some $v\in R$ with $uv =1$.
\end{definition}

Are left units always right units? We know this to be true in commutative rings and $M_{n\times n}(k)$, but if we consider $GL(\bR^\bN)$, with the basis $\{e_1,e_2,\ldots\}$, then the linear map which sends each $e_i$ to $e_{i+1}$ is injective and has a left inverse, but is not surjective so has no right inverse. But, if $u$ has both a right inverse \textbf{and} a left inverse, then these will always be the same. The group of two-sided units is called $R^\times$.

\section{Matrix Lie groups}

\end{document}